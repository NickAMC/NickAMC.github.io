% Options for packages loaded elsewhere
\PassOptionsToPackage{unicode}{hyperref}
\PassOptionsToPackage{hyphens}{url}
\PassOptionsToPackage{dvipsnames,svgnames,x11names}{xcolor}
%
\documentclass[
  11pt,
]{article}

\usepackage{amsmath,amssymb}
\usepackage{iftex}
\ifPDFTeX
  \usepackage[T1]{fontenc}
  \usepackage[utf8]{inputenc}
  \usepackage{textcomp} % provide euro and other symbols
\else % if luatex or xetex
  \usepackage{unicode-math}
  \defaultfontfeatures{Scale=MatchLowercase}
  \defaultfontfeatures[\rmfamily]{Ligatures=TeX,Scale=1}
\fi
\usepackage{lmodern}
\ifPDFTeX\else  
    % xetex/luatex font selection
\fi
% Use upquote if available, for straight quotes in verbatim environments
\IfFileExists{upquote.sty}{\usepackage{upquote}}{}
\IfFileExists{microtype.sty}{% use microtype if available
  \usepackage[]{microtype}
  \UseMicrotypeSet[protrusion]{basicmath} % disable protrusion for tt fonts
}{}
\makeatletter
\@ifundefined{KOMAClassName}{% if non-KOMA class
  \IfFileExists{parskip.sty}{%
    \usepackage{parskip}
  }{% else
    \setlength{\parindent}{0pt}
    \setlength{\parskip}{6pt plus 2pt minus 1pt}}
}{% if KOMA class
  \KOMAoptions{parskip=half}}
\makeatother
\usepackage{xcolor}
\usepackage[margin=1in]{geometry}
\setlength{\emergencystretch}{3em} % prevent overfull lines
\setcounter{secnumdepth}{-\maxdimen} % remove section numbering
% Make \paragraph and \subparagraph free-standing
\ifx\paragraph\undefined\else
  \let\oldparagraph\paragraph
  \renewcommand{\paragraph}[1]{\oldparagraph{#1}\mbox{}}
\fi
\ifx\subparagraph\undefined\else
  \let\oldsubparagraph\subparagraph
  \renewcommand{\subparagraph}[1]{\oldsubparagraph{#1}\mbox{}}
\fi

\usepackage{color}
\usepackage{fancyvrb}
\newcommand{\VerbBar}{|}
\newcommand{\VERB}{\Verb[commandchars=\\\{\}]}
\DefineVerbatimEnvironment{Highlighting}{Verbatim}{commandchars=\\\{\}}
% Add ',fontsize=\small' for more characters per line
\usepackage{framed}
\definecolor{shadecolor}{RGB}{241,243,245}
\newenvironment{Shaded}{\begin{snugshade}}{\end{snugshade}}
\newcommand{\AlertTok}[1]{\textcolor[rgb]{0.68,0.00,0.00}{#1}}
\newcommand{\AnnotationTok}[1]{\textcolor[rgb]{0.37,0.37,0.37}{#1}}
\newcommand{\AttributeTok}[1]{\textcolor[rgb]{0.40,0.45,0.13}{#1}}
\newcommand{\BaseNTok}[1]{\textcolor[rgb]{0.68,0.00,0.00}{#1}}
\newcommand{\BuiltInTok}[1]{\textcolor[rgb]{0.00,0.23,0.31}{#1}}
\newcommand{\CharTok}[1]{\textcolor[rgb]{0.13,0.47,0.30}{#1}}
\newcommand{\CommentTok}[1]{\textcolor[rgb]{0.37,0.37,0.37}{#1}}
\newcommand{\CommentVarTok}[1]{\textcolor[rgb]{0.37,0.37,0.37}{\textit{#1}}}
\newcommand{\ConstantTok}[1]{\textcolor[rgb]{0.56,0.35,0.01}{#1}}
\newcommand{\ControlFlowTok}[1]{\textcolor[rgb]{0.00,0.23,0.31}{#1}}
\newcommand{\DataTypeTok}[1]{\textcolor[rgb]{0.68,0.00,0.00}{#1}}
\newcommand{\DecValTok}[1]{\textcolor[rgb]{0.68,0.00,0.00}{#1}}
\newcommand{\DocumentationTok}[1]{\textcolor[rgb]{0.37,0.37,0.37}{\textit{#1}}}
\newcommand{\ErrorTok}[1]{\textcolor[rgb]{0.68,0.00,0.00}{#1}}
\newcommand{\ExtensionTok}[1]{\textcolor[rgb]{0.00,0.23,0.31}{#1}}
\newcommand{\FloatTok}[1]{\textcolor[rgb]{0.68,0.00,0.00}{#1}}
\newcommand{\FunctionTok}[1]{\textcolor[rgb]{0.28,0.35,0.67}{#1}}
\newcommand{\ImportTok}[1]{\textcolor[rgb]{0.00,0.46,0.62}{#1}}
\newcommand{\InformationTok}[1]{\textcolor[rgb]{0.37,0.37,0.37}{#1}}
\newcommand{\KeywordTok}[1]{\textcolor[rgb]{0.00,0.23,0.31}{#1}}
\newcommand{\NormalTok}[1]{\textcolor[rgb]{0.00,0.23,0.31}{#1}}
\newcommand{\OperatorTok}[1]{\textcolor[rgb]{0.37,0.37,0.37}{#1}}
\newcommand{\OtherTok}[1]{\textcolor[rgb]{0.00,0.23,0.31}{#1}}
\newcommand{\PreprocessorTok}[1]{\textcolor[rgb]{0.68,0.00,0.00}{#1}}
\newcommand{\RegionMarkerTok}[1]{\textcolor[rgb]{0.00,0.23,0.31}{#1}}
\newcommand{\SpecialCharTok}[1]{\textcolor[rgb]{0.37,0.37,0.37}{#1}}
\newcommand{\SpecialStringTok}[1]{\textcolor[rgb]{0.13,0.47,0.30}{#1}}
\newcommand{\StringTok}[1]{\textcolor[rgb]{0.13,0.47,0.30}{#1}}
\newcommand{\VariableTok}[1]{\textcolor[rgb]{0.07,0.07,0.07}{#1}}
\newcommand{\VerbatimStringTok}[1]{\textcolor[rgb]{0.13,0.47,0.30}{#1}}
\newcommand{\WarningTok}[1]{\textcolor[rgb]{0.37,0.37,0.37}{\textit{#1}}}

\providecommand{\tightlist}{%
  \setlength{\itemsep}{0pt}\setlength{\parskip}{0pt}}\usepackage{longtable,booktabs,array}
\usepackage{calc} % for calculating minipage widths
% Correct order of tables after \paragraph or \subparagraph
\usepackage{etoolbox}
\makeatletter
\patchcmd\longtable{\par}{\if@noskipsec\mbox{}\fi\par}{}{}
\makeatother
% Allow footnotes in longtable head/foot
\IfFileExists{footnotehyper.sty}{\usepackage{footnotehyper}}{\usepackage{footnote}}
\makesavenoteenv{longtable}
\usepackage{graphicx}
\makeatletter
\def\maxwidth{\ifdim\Gin@nat@width>\linewidth\linewidth\else\Gin@nat@width\fi}
\def\maxheight{\ifdim\Gin@nat@height>\textheight\textheight\else\Gin@nat@height\fi}
\makeatother
% Scale images if necessary, so that they will not overflow the page
% margins by default, and it is still possible to overwrite the defaults
% using explicit options in \includegraphics[width, height, ...]{}
\setkeys{Gin}{width=\maxwidth,height=\maxheight,keepaspectratio}
% Set default figure placement to htbp
\makeatletter
\def\fps@figure{htbp}
\makeatother

\usepackage{fvextra}
\DefineVerbatimEnvironment{Highlighting}{Verbatim}{breaklines,commandchars=\\\{\}}
\DefineVerbatimEnvironment{OutputCode}{Verbatim}{breaklines,commandchars=\\\{\}}
\fvset{breaksymbolleft={}, breakindent=1em}
\makeatletter
\@ifpackageloaded{caption}{}{\usepackage{caption}}
\AtBeginDocument{%
\ifdefined\contentsname
  \renewcommand*\contentsname{Table of contents}
\else
  \newcommand\contentsname{Table of contents}
\fi
\ifdefined\listfigurename
  \renewcommand*\listfigurename{List of Figures}
\else
  \newcommand\listfigurename{List of Figures}
\fi
\ifdefined\listtablename
  \renewcommand*\listtablename{List of Tables}
\else
  \newcommand\listtablename{List of Tables}
\fi
\ifdefined\figurename
  \renewcommand*\figurename{Figure}
\else
  \newcommand\figurename{Figure}
\fi
\ifdefined\tablename
  \renewcommand*\tablename{Table}
\else
  \newcommand\tablename{Table}
\fi
}
\@ifpackageloaded{float}{}{\usepackage{float}}
\floatstyle{ruled}
\@ifundefined{c@chapter}{\newfloat{codelisting}{h}{lop}}{\newfloat{codelisting}{h}{lop}[chapter]}
\floatname{codelisting}{Listing}
\newcommand*\listoflistings{\listof{codelisting}{List of Listings}}
\makeatother
\makeatletter
\makeatother
\makeatletter
\@ifpackageloaded{caption}{}{\usepackage{caption}}
\@ifpackageloaded{subcaption}{}{\usepackage{subcaption}}
\makeatother
\ifLuaTeX
  \usepackage{selnolig}  % disable illegal ligatures
\fi
\usepackage{bookmark}

\IfFileExists{xurl.sty}{\usepackage{xurl}}{} % add URL line breaks if available
\urlstyle{same} % disable monospaced font for URLs
\hypersetup{
  pdftitle={Homework 3},
  pdfauthor={Nick Climaco},
  colorlinks=true,
  linkcolor={blue},
  filecolor={Maroon},
  citecolor={Blue},
  urlcolor={Blue},
  pdfcreator={LaTeX via pandoc}}

\title{Homework 3}
\author{Nick Climaco}
\date{February 12, 2024}

\begin{document}
\maketitle

\RecustomVerbatimEnvironment{verbatim}{Verbatim}{
  showspaces = false,
  showtabs = false,
  breaksymbolleft={},
  breaklines
}

\renewcommand*\contentsname{Table of contents}
{
\hypersetup{linkcolor=}
\setcounter{tocdepth}{3}
\tableofcontents
}
\newpage

\section{Chapter 5: The Forecaster's
Toolbox}\label{chapter-5-the-forecasters-toolbox}

\subsection{Exercise 1}\label{exercise-1}

Produce forecasts for the following series using whichever of NAIVE(y),
SNAIVE(y) or RW(y \textasciitilde{} drift()) is more appropriate in each
case:

\begin{itemize}
\item
  Australian Population (global\_economy)
\item
  Bricks (aus\_production)
\item
  NSW Lambs (aus\_livestock)
\item
  Household wealth (hh\_budget).
\item
  Australian takeaway food turnover (aus\_retail).
\end{itemize}

\begin{Shaded}
\begin{Highlighting}[]
\NormalTok{df\_aus\_prod }\OperatorTok{=}\NormalTok{ pd.read\_csv(}\StringTok{"../rdata/global\_economy.csv"}\NormalTok{, parse\_dates}\OperatorTok{=}\NormalTok{[}\StringTok{"Year"}\NormalTok{], index\_col}\OperatorTok{=}\NormalTok{[}\StringTok{\textquotesingle{}Year\textquotesingle{}}\NormalTok{])}
\NormalTok{df\_aus\_prod}
\end{Highlighting}
\end{Shaded}

\begin{longtable}[]{@{}llllllll@{}}
\toprule\noalign{}
& Unnamed: 0 & Country & Code & ... & Imports & Exports & Population \\
Year & & & & & & & \\
\midrule\noalign{}
\endhead
\bottomrule\noalign{}
\endlastfoot
1960-01-01 & 1 & Afghanistan & AFG & ... & 7.024793 & 4.132233 &
8996351.0 \\
1961-01-01 & 2 & Afghanistan & AFG & ... & 8.097166 & 4.453443 &
9166764.0 \\
1962-01-01 & 3 & Afghanistan & AFG & ... & 9.349593 & 4.878051 &
9345868.0 \\
1963-01-01 & 4 & Afghanistan & AFG & ... & 16.863910 & 9.171601 &
9533954.0 \\
1964-01-01 & 5 & Afghanistan & AFG & ... & 18.055555 & 8.888893 &
9731361.0 \\
... & ... & ... & ... & ... & ... & ... & ... \\
2013-01-01 & 15146 & Zimbabwe & ZWE & ... & 36.668735 & 21.987759 &
15054506.0 \\
2014-01-01 & 15147 & Zimbabwe & ZWE & ... & 33.741470 & 20.930146 &
15411675.0 \\
2015-01-01 & 15148 & Zimbabwe & ZWE & ... & 37.588635 & 19.160176 &
15777451.0 \\
2016-01-01 & 15149 & Zimbabwe & ZWE & ... & 31.275493 & 19.943532 &
16150362.0 \\
2017-01-01 & 15150 & Zimbabwe & ZWE & ... & 30.370273 & 19.658023 &
16529904.0 \\
\end{longtable}

\subsection{Exercise 2}\label{exercise-2}

Use the Facebook stock price (data set gafa\_stock) to do the following:

\begin{itemize}
\item
  Produce a time plot of the series.
\item
  Produce forecasts using the drift method and plot them.
\item
  Show that the forecasts are identical to extending the line drawn
  between the first and last observations.
\item
  Try using some of the other benchmark functions to forecast the same
  data set. Which do you think is best? Why?
\end{itemize}

\subsection{Exercise 3}\label{exercise-3}

Apply a seasonal naïve method to the quarterly Australian beer
production data from 1992. Check if the residuals look like white noise,
and plot the forecasts. The following code will help.

\begin{Shaded}
\begin{Highlighting}[]
\NormalTok{\# Extract data of interest}
\NormalTok{recent\_production \textless{}{-} aus\_production |\textgreater{}}
\NormalTok{  filter(year(Quarter) \textgreater{}= 1992)}

\NormalTok{\# Define and estimate a model}
\NormalTok{fit \textless{}{-} recent\_production |\textgreater{} model(SNAIVE(Beer))}
\NormalTok{\#Look at the residuals}

\NormalTok{fit |\textgreater{} gg\_tsresiduals()}

\NormalTok{\# Look a some forecasts}
\NormalTok{fit |\textgreater{} forecast() |\textgreater{} autoplot(recent\_production)}
\end{Highlighting}
\end{Shaded}

What do you conclude?

\subsection{Exercise 4}\label{exercise-4}

Repeat the previous exercise using the Australian Exports series from
global\_economy and the Bricks series from aus\_production. Use
whichever of NAIVE() or SNAIVE() is more appropriate in each case

\subsection{Exercise 7}\label{exercise-7}

For your retail time series (from Exercise 7 in Section 2.10):

\begin{enumerate}
\def\labelenumi{\alph{enumi}.}
\tightlist
\item
  Create a training dataset consisting of observations before 2011 using
\end{enumerate}

\begin{Shaded}
\begin{Highlighting}[]
\NormalTok{myseries\_train \textless{}{-} myseries |\textgreater{}}
\NormalTok{  filter(year(Month) \textless{} 2011)}
\end{Highlighting}
\end{Shaded}

\begin{enumerate}
\def\labelenumi{\alph{enumi}.}
\setcounter{enumi}{1}
\tightlist
\item
  Check that your data have been split appropriately by producing the
  following plot.
\end{enumerate}

\begin{Shaded}
\begin{Highlighting}[]
\NormalTok{autoplot(myseries, Turnover) +}
\NormalTok{  autolayer(myseries\_train, Turnover, colour = "red")}
\end{Highlighting}
\end{Shaded}

\begin{enumerate}
\def\labelenumi{\alph{enumi}.}
\setcounter{enumi}{2}
\tightlist
\item
  Fit a seasonal naïve model using SNAIVE() applied to your training
  data (myseries\_train).
\end{enumerate}

\begin{Shaded}
\begin{Highlighting}[]
\NormalTok{fit \textless{}{-} myseries\_train |\textgreater{}}
\NormalTok{  model(SNAIVE())}
\end{Highlighting}
\end{Shaded}

\begin{enumerate}
\def\labelenumi{\alph{enumi}.}
\setcounter{enumi}{3}
\tightlist
\item
  Check the residuals.
\end{enumerate}

\begin{Shaded}
\begin{Highlighting}[]
\NormalTok{fit |\textgreater{} gg\_tsresiduals()}
\end{Highlighting}
\end{Shaded}

Do the residuals appear to be uncorrelated and normally distributed?

e.Produce forecasts for the test data

\begin{Shaded}
\begin{Highlighting}[]
\NormalTok{fc \textless{}{-} fit |\textgreater{}}
\NormalTok{  forecast(new\_data = anti\_join(myseries, myseries\_train))}
\NormalTok{fc |\textgreater{} autoplot(myseries)}
\end{Highlighting}
\end{Shaded}

\begin{enumerate}
\def\labelenumi{\alph{enumi}.}
\setcounter{enumi}{5}
\tightlist
\item
  Compare the accuracy of your forecasts against the actual values.
\end{enumerate}

\begin{Shaded}
\begin{Highlighting}[]
\NormalTok{fit |\textgreater{} accuracy()}
\NormalTok{fc |\textgreater{} accuracy(myseries)}
\end{Highlighting}
\end{Shaded}

g.How sensitive are the accuracy measures to the amount of training data
used?



\end{document}
